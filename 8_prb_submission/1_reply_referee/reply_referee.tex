%\documentclass[aps,prb,amsmath,twocolumn,amssymb,titlepage]{revtex4-1}
\documentclass[10pt]{revtex4-1}
\usepackage[utf8]{inputenc}
\usepackage[T1]{fontenc}
\usepackage{graphicx}
\usepackage{nicefrac}
\usepackage{amsfonts}
\usepackage{amssymb}
\usepackage{amsmath} 
\usepackage{subfigure}
\usepackage{multirow} 
\usepackage{tabularx} 
\usepackage{array}
\usepackage{units}
\usepackage{tensor} 
\usepackage{braket}
\usepackage{bm}
\usepackage{hyperref}
\usepackage[dvipsnames]{xcolor}
\usepackage[resetlabels, labeled]{multibib}
%\newcites{supp}{References}

\renewcommand{\Re}{\operatorname{{\mathrm Re}}}
\renewcommand{\Im}{\operatorname{{\mathrm Im}}}
\newcommand{\tr}{\operatorname{{\mathrm tr}}} 

\newcommand{\hc}{^{\dagger}}
\newcommand{\ad}{\operatorname{{\mathrm ad}}}
\newcommand{\adn}{\ad_{\hat n}}
\newcommand{\adx}{\ad_{\hat x}}
\newcommand{\adX}[1][]{\ad_{\hat {\mathbf{X}}_{#1}}}
\newcommand{\adp}{\ad_{\hat p}}
\newcommand{\calD}{\mathcal{D}}    %Caligraphic D
\newcommand{\inbk}[1]{\left[ #1 \right]}
\newcommand{\inbr}[1]{\left\{ #1 \right\}}
\newcommand{\inp}[1]{\left( #1 \right)}
\newcommand{\pd}{\partial}
\newcommand{\pdf}[3]{\frac{\pd^{#1} #2}{\pd #3^{#1}}} %Partial derivative
\newcommand{\fdf}[3][]{\frac{\delta^{#1} #2}{\delta #3^{#1}}} %Functional derivative

\newcounter{quoter}
\setcounter{quoter}{0}
\renewcommand*{\thequoter}{%
  %\textbf{%
  %  \ifnum\value{custom}<1000 0\fi
  %  \ifnum\value{custom}<100 0\fi
  %  \ifnum\value{custom}<10 0\fi
  %  \arabic{custom}%
  \textbf{[\arabic{quoter}]}
  }%

\newcommand{\genquote}[1]{\refstepcounter{quoter} \label{#1} \ref{#1}}
\newcommand{\citequote}[1]{\ref{#1}}

\definecolor{antiquefuchsia}{rgb}{0.57, 0.36, 0.51}

%\stepcounter{quote}
%\refstepcounter{quote}
 
\begin{document}
\section{Reply to the first referee }
\section{Reply to the second referee}
Dear referee,

we appreciate the valuable insigths on how to improve our article that you have provided us. Below we address each of the points you emphasized separately, as they were written in your reply. We quote the issues raised by you in purple, and highlight in boldface the points that we thought were particularly relevant. 

\subsection{Major}

\textcolor{antiquefuchsia}{I am not sure about the novelty and generality of their method. For
example, \textbf{their pipeline cannot work for complex experimental physical
systems, where the Hamiltonian could be unknown}\genquote{pipeline_not_work}. Moreover, the authors
state that it is sufficient to use the input data in real space to
predict the topological phase with high accuracy. \textbf{It is not clear to
me what "sufficient" and "high accuracy" mean}\genquote{sufficient_accuracy}, and the advantages of
using the input data in real space instead of the Hamiltonian in
wavevector space. \textbf{I expect the evidence to indicate that their method
is superior or comparable to any other method}\genquote{superior_comparable}. It is worth mentioning
that the author already discussed \textbf{the motivations for developing a
data-driven approach based on real space}\genquote{motif_real_space} in the "Learning topological
phases from real space data" section. However, these arguments should
be pointed carefully in the "Introduction" section.}

\vspace{0.25cm}
\citequote{pipeline_not_work} Mention the possibility of using incomplete information with eigenvector ensembling.

\citequote{sufficient_accuracy} ...

\citequote{superior_comparable} Mention trends in model explainability as well as performance scores.  

\citequote{motif_real_space}We agree that the motivation for using real space data to study topological phase transitions should be placed in the introduction. We have therefore moved the first paragraph of the section "Learning topological phases from real space data" to the introduction.  

\vspace{0.5cm}
\textcolor{antiquefuchsia}{I am not sure which crucial problems their algorithm can solve that
was not possible before with machine learning. Furthermore, along with
the emerging research of unsupervised learning methods in realizing
the topological phases, \textbf{why should the supervised learning method be
focused in this context?}\genquote{unsupervised_learning} In the present stage of this manuscript,
where I do not see the proposal's advantages, I would prefer the
\textbf{unsupervised approaches that could grasp information from a given
system without knowing their phases}\genquote{unsupervised_approaches}. In my opinion, a proper intrinsic
route for understanding physical systems without much information on
the dynamics of the system will lead to significant future
developments and a better understanding of physic systems.}

\vspace{0.25cm}
\citequote{unsupervised_learning} Mention semi-supervised learning, model explainability.

\citequote{unsupervised_approaches} Mention Hamiltonian compression, of unsupervised learning through dimensional reduction.

\vspace{0.5cm}
\textcolor{antiquefuchsia}{The details of \textbf{the eigenvector ensembling algorithm should be
addressed in the main text}\genquote{eigenvector_ensembling_alg} instead of supplemental material. What is
the specific algorithm used in the paper to train on eigenvectors? I
think it is important even if the readers are not familiar with some
ML techniques. Some readers in physics may find it difficult to
understand some ML technical terms such as bootstrapping, training and
validation, test sets, etc. In addition, \textbf{I expect proof of what
"physics" their method captures and why the eigenvector can play an
important role in characterizing the topological phase}\genquote{eigenvector_physics}. If this
physical interpretability is not mentioned, it is very difficult to
see the contribution of the method in physics.}

\vspace{0.25cm}
\citequote{eigenvector_ensembling_alg} We have moved the section "The eigenvector ensembling algorithm" to the main paper. The issue of specifying the particular learning algorithms used was also addressed.

\citequote{eigenvector_physics} One of the strongest points of the paper is to introduce model explainability tools from machine learning in physics. The need to use eigenvectors is that they encode all the physical information from a system. The paper paves the way for new theoretical investigation on the relations between the shannon entropy of eigenvectors and topological phase transitions. 

\vspace{0.5cm}
\textcolor{antiquefuchsia}{The authors made efforts to analyze how the algorithm was able to
recover the Hamiltonians' global property in the "Information Entropy
Signatures" section. The authors state that learning topological
phases from local real-space data in bulk is still possible even for
small subsets of lattice sites, then refer us to the section "Learning
topological phases from real space data" in the Supplementary
Material. The authors mention on page 4 of the Supplementary Material
that \textbf{"key topological information can be said to be localized on a few
lattice sites," which is a particularly interesting statement to me}\genquote{key_top}.
However, I fail to understand the physical insights behind it. Without
the proper explanation, \textbf{it is difficult to see the effectiveness of
their method or verify the method with a more complex model}\genquote{more_complex_model} instead of
the simple SSH form.}

\vspace{0.25cm}
\citequote{key_top} Move this sentence to main paper.

\citequote{more_complex_model} Add FFT of entropy signatures? Mention entropic uncertainty? Talk about information gain? Mention that interpretability in ML is often challenging and a fast evolving field. Furthermore, eigenvector ensembling can always be used as a preprocessing step to other ML models, such as in semi supervised learning tasks. 

\subsection{Minor}
\end{document}
